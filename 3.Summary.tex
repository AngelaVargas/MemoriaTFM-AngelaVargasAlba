
%%%%%%%%%%%%%%%%%%%%%%%%%%%%%%%%%%%%%%%%%%%%%%%%%%%%%%%%%%%%%%%%%%%%%%%%%%%%%%%%
%%%% Summary

\chapter*{Resumen}
%\addcontentsline{toc}{chapter}{Resumen} % si queremos que aparezca en el índice
\markboth{RESUMEN}{RESUMEN} 

Este proyecto consiste en una evaluación y análisis del impacto que tienen los malos hábitos en el mundo de la programación. En concreto, trata de analizar el efecto que producen los ``bad smells'' en el desarollo de las habilidades del Pensamiento Computacional.  

Para su desarrollo, nos hemos basado en la herramienta Dr. Scratch, una aplicación web de software libre que permite analizar proyectos diseñados con Scratch -lenguaje de programación orientado a la educación- y obtener una evaluación sobre diferentes aspectos relacionados con el Pensamiento Computacional.

El objetivo final del proyecto es la implementación de un nuevo modelo de evaluación en Dr. Scratch que permita concienciar y prevenir sobre el uso de ``bad smells'' en la programación con Scratch. 

Para llevar a cabo este proceso, han sido necesarias diversas fases de trabajo con diferentes tecnologías implicadas. En una fase inicial, fue necesaria una actualización de la herramienta Dr. Scratch. Para ello se utilizaron tecnologías relacionadas con la programación web y el entorno de producción, tales como Django, MySQL, Microsoft Azure o Google Cloud Platform. Durante la segunda fase del proyecto se llevó a cabo un análisis exhaustivo sobre ``bad smells'. Para este procedimiento se utilizó Jupyter Notebook, una tecnología propia del análisis de datos. Para el diseño web del nuevo modelo descrito anteriormente, lo que constituye la tercera fase, se utilizaron tecnologías como HTML, CSS y Bootstrap. Finalmente, para comprobar la efectivad del proyecto, se diseñó e implementó un experimento de evaluación con diferentes profesores. Para la consecución de esta última fase, se utilizaron las mismas tecnologías que para el tratamiento de datos, junto con la propia herramienta Dr. Scratch.


%%%%%%%%%%%%%%%%%%%%%%%%%%%%%%%%%%%%%%%%%%%%%%%%%%%%%%%%%%%%%%%%%%%%%%%%%%%%%%%%
%%%% Resumen en inglés

\chapter*{Summary}
%\addcontentsline{toc}{chapter}{Summary} % si queremos que aparezca en el índice
\markboth{SUMMARY}{SUMMARY} % encabezado

Here comes a translation of the ``Resumen'' into English. 
Please, double check it for correct grammar and spelling.
As it is the translation of the ``Resumen'', which is supposed to be written at the end, this as well should be filled out just before submitting.
