

%%%%%%%%%%%%%%%%%%%%%%%%%%%%%%%%%%%%%%%%%%%%%%%%%%%%%%%%%%%%%%%%%%%%%%%%%%%%%%%%
%%%% Summary in English

\chapter*{Summary}
\addcontentsline{toc}{chapter}{Summary} 
\markboth{SUMMARY}{SUMMARY} 

This project consists of the evaluation and analysis of the impact which bad habits have in the programming area. In particular, in this work we want to analyze the effect of ``bad smells'' in the development of the Computational Thinking skills. 

For its development, we have based on the Dr. Scratch tool, a free software web application which allows the analysis of projects designed with Scratch -programming language oriented to education- and to obtain an assessment about different aspects related to the Computational Thinking.

The final objective of the project is the implementation of a new assessment model in Dr. Scratch which allows to raise awareness and prevent about the use of ``bad smells'' in programming with Scratch.

In order to carry out this process, several phases of work have been necessary, with different technologies involved. In an initial phase, it was needed an update of the Dr. Scratch tool. The technologies used for that were related to web programming and the cloud production environment, such as Django, MySQL, Microsoft Azure or Google Cloud Platform, among others. During the second phase, we carried out an exhaustive analysis about ``bad smells''. For this procedure, we used Jupyter Notebook, an appropriate technology for the data analysis. For the web design of the new model described previously, which constitutes the third phase, technologies such as HTML, CSS and Bootstrap were used. Finally, in order to verify the effectiveness of the project, we designed and implemented an assessment experiment with different teachers. To achieve this last phase, Google Forms together with the Dr. Scratch tool were used.


%%%%%%%%%%%%%%%%%%%%%%%%%%%%%%%%%%%%%%%%%%%%%%%%%%%%%%%%%%%%%%%%%%%%%%%%%%%%%%%%
%%%% Summary in Spanish

\chapter*{Resumen}
\addcontentsline{toc}{chapter}{Resumen} 
\markboth{RESUMEN}{RESUMEN} 

Este proyecto consiste en una evaluación y análisis del impacto que tienen los malos hábitos en el mundo de la programación. En concreto, trata de analizar el efecto que producen los ``bad smells'' en el desarollo de las habilidades del Pensamiento Computacional.  

Para su desarrollo, nos hemos basado en la herramienta Dr. Scratch, una aplicación web de software libre que permite analizar proyectos diseñados con Scratch -lenguaje de programación orientado a la educación- y obtener una evaluación sobre diferentes aspectos relacionados con el Pensamiento Computacional.

El objetivo final del proyecto es la implementación de un nuevo modelo de evaluación en Dr. Scratch que permita concienciar y prevenir sobre el uso de ``bad smells'' en la programación de proyectos Scratch. 

Para llevar a cabo este proceso, han sido necesarias diversas fases de trabajo con diferentes tecnologías implicadas. En una fase inicial, fue necesaria una actualización de la herramienta Dr. Scratch. Para ello se utilizaron tecnologías relacionadas con la programación web y el entorno de producción en la nube, tales como Django, MySQL, Microsoft Azure o Google Cloud Platform. Durante la segunda fase del proyecto se llevó a cabo un análisis exhaustivo sobre ``bad smells'. Para este procedimiento se utilizó Jupyter Notebook, una tecnología propia del análisis de datos. Para el diseño web del nuevo modelo descrito anteriormente, lo que constituye la tercera fase, se utilizaron tecnologías como HTML, CSS y Bootstrap. Finalmente, para comprobar la efectivad del proyecto, se diseñó e implementó un experimento de evaluación con diferentes profesores. Para la consecución de esta última fase, se utilizó Google Forms junto con la propia herramienta Dr. Scratch.
